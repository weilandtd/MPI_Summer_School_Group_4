\documentclass[hyperref]{labbook}

%\usepackage[utf8]{inputenc}
\usepackage[english]{babel}

\usepackage[top=1in, bottom=1in, left=1in, right=1in]{geometry}

\usepackage{amsmath}
\usepackage{graphicx}
\usepackage{epstopdf}
\usepackage{float}
\usepackage{listings}
\usepackage{color}
\usepackage{hyperref}
\usepackage{tikz}

\numberwithin{equation}{experiment}
\numberwithin{table}{experiment}
\numberwithin{figure}{experiment}

%Formatting of the Code Parts 
\definecolor{maroon}{rgb}{0.5,0,0}
\definecolor{darkgreen}{rgb}{0,0.5,0}
\lstdefinelanguage{XML}
{
  basicstyle=\ttfamily\footnotesize,
  morestring=[b]",
  moredelim=[s][\bfseries\color{cyan}]{<}{\  },
  moredelim=[s][\bfseries\color{cyan}]{</}{>},
  moredelim=[l][\bfseries\color{cyan}]{/>},
  moredelim=[l][\bfseries\color{cyan}]{>},
  morecomment=[s]{<?}{?>},
  morecomment=[s]{<!--}{-->},
  commentstyle=\color{orange},
  stringstyle=\color{red},
  identifierstyle=\color{darkgreen}
}


\begin{document}

\numberwithin{lstlisting}{experiment}
\labday{24.07.2016}
\experiment{Dynamic systems modeling}
System: a potential source o data
\begin{itemize}
\item{\color{red} boundary }
\item inputs / outputs
\end{itemize}
Experiments: Extracting data from a system 
\begin{itemize}
\item apply condition to input and observe outputs
(Obeservability / Controlability)
\end{itemize}
Model: of a system {\color{red} and an experiment is anything} to which exp can be applied.
This is results in an experimental frame of the model. A model in the general case doesn't need to be mathematical or computational. No model is valid for all experiments but the system itself.
The purpose of modeling is to simply. "All models are wrong but some of them are useful" \\[1em]
Simulation: Perform the experiment on the model. \\[1em]
Continuity: 
\begin{itemize}
\item Discrete and Deterministic  (Molecular Dynamics)
\item {\color{red} Continuous  and Deterministic (Partial differential equations)}
\item Discrete and Stochastic  (Agent based models)
\item Continuous  and Stochastic (Stochastic differential equations) 
\end{itemize}

The continuum limit assumes the changes of a single molecular does not matter for the overall concentration in the region of the length scale  $\lambda$. 
The length scale $L$ field gradients are established  \\
\begin{equation}
\mathrm{Kn} := \frac{\lambda}{L} 
\end{equation}•
If $\mathrm{Kn} << 1 $ the continuum assumption is justified. \\[1em]
Extensive quantities: depend on the system size.(Mass, Volume, molecules \dots) \\
Intensive quantities: do not depend on the system size (Temperature, concentration \dots). \\[1em]
Differential equations are always expressed in terms of intensive quantities in simulation needs to use extensive quantities. This is not necessary but helpful to design the simulation experiment properly. \newpage
Dimensional Analysis: 
\begin{itemize}
\item dependence of variables (Which are the important variables?)
\item orders of magnitude
\item Model - System similitude (Where show model and system same dynamics?)
\end{itemize}
Dim:   Mass, length, time, force,  \dots\\
Units: kg, lb \dots \\[1em]
Theorem:\\
All dimensions are power series of 6 basic dimensions.
\begin{itemize}
\item mass M
\item length L 
\item time T 
\item temperature $\theta$
\item charge C
\item resistance R
\end{itemize}
Example: $force = M^1L^1T^{-1}$\\[1em]
Buckingham Theorem:\\
If n is the number of basic dimensions  and p variables of a system. The system is completely described by p-n dimensionless groupings. \\
Example:\\
A force acting on a point mass in vacuum: \\
$n = 3 (\mathrm{M,L,T})$\\
$p = 4 (\mathrm{m,F,t,v})$\\
Therefore, the system is described by a single dimensionless quantity.\\[1em]
All physical equations are dimensionally homogeneous: $A + B = C $. The dimensions of A, B and C needs to be the same. \\[1em]
Taylor (1979) proposed and algorithm to find the dimensionless groupings of a system. \\
Example:\\
Shear stress on red blood cell traveling with a velocity $u$ though an artery. If the shear stress $\tau$ is to hight the blood cell would be destroyed. The shear stress then depends on the distance from the wall $h$, the viscosity $\eta$, the density $\rho$ and the velocity $u$.\\[1em]
$n = 3 (M,L,T)$\\
$p = 4 (\eta,\rho,h,\tau,u)$\\

\begin{equation}
\begin{matrix}
          & M & L & T \\
\hline
h       & 0  & 1 & 0 \\
u       & 0  & 1 & -1 \\
\eta & 1 & -1 & -1 \\
\rho & 1  & -3 & 0 \\
\tau & 1  & -1 & -2 \\
\end{matrix} \rightarrow
\begin{matrix}
          & M & L & T \\
\hline
h       & 0  & 1 & 0 \\
u       & 0  & 1 & -1 \\
\eta & 0 &  2 & -1 \\
\rho & 1  & -3 & 0 \\
\tau & 0  & 2 & -2 \\
\end{matrix}
\rightarrow
\begin{matrix}
           & L & T \\
\hline
h         & 1 & 0 \\
u         & 1 & -1 \\
\eta/\rho u &  1 &0 \\
\tau/\rho u^2  &0 & 0 \\
\end{matrix}
\rightarrow
\begin{matrix}
           & L  \\
\hline
\eta/\rho u h&  0 \\
\tau/\rho u^2  &0 \\
\end{matrix}
\end{equation}
The dimensionless numbers are then $\Pi_1 = \frac{\eta}{\rho u h}$ and $\Pi_1 = \frac{\tau}{\rho u^2}$.\\
We also know that: $\Pi_1 = f(\Pi_2 )$ this helps us to design an experiment because only the dimensionless numbers need to be varied to understand the function the relates  $\Pi_1$ and $\Pi_2$ \\[2em]
Modeling Dynamics:\\[0.5em]
1) Define the system boundaries and the input and outputs\\
2) Identify reservoirs (for any quantity) of relevant time scales \\
3) Formulate equations for the flows to the reservoirs. \\All flows have the form $flow = f(activating level - inhibiting level)$. \\
4) Formulate the balance equations for the reservoirs. $\frac{d}{dt}level = \sum inflows - \sum outflows $
5) Simplify the system of equations\\
6) Solving the systems will give $level(t)$\\
7) Identify unknown parameters based on the given data\\
8) {\color{red} Validate the model}. Prove that the model can predict another experiment!\\[1em]
\experiment{Recap of vector calculus}

\experiment{Spatiotemporal models}
Control volumes:\\
Discretization of space into finite small volumes. Considering that the control volumes can vary with time the rate of change of the intensive variable $f$. For being the density $f = m/V$:
\begin{equation}
\frac{df}{dt} = \frac{1}{V}\frac{dm}{dt} + \frac{m}{V}\frac{1}{V}\left( -\frac{dV}{dt}\right)
\end{equation}•\\[1em]
Fixes control volumes / Eulerian :\\
When using fixed control volumes the flow between the control volumes $V_i$ is define by the field derivative: 
 \begin{equation}
\frac{\partial f(x,t)}{\partial t }|_{x=const} = D\nabla^2f - \nabla \cdot (f \cdot \mathbf{v}(\mathbf{r},t)) + k(f)
\end{equation}
Co-moving control volume / Lagrangian: \\
When considering moving volumes the flow of the intensive quantity is given by the material derivative: 
\begin{equation}
\frac{\mathrm{d}f}{\mathrm{d}t} = \frac{\partial}{\partial t} \left[f(\mathbf{r}(r), t ) \right] = \nabla f \cdot \frac{\partial  \mathbf{r}(t)}{\partial t} + \frac{\partial f }{\partial t}
\end{equation}\\[1em]
For considering mass conservation: 
\begin{equation}
m = \int f \mathrm{d}V  
\end{equation}•
\begin{equation}
\frac{\mathrm{d}m}{\mathrm{d}t} = 0 
\end{equation}
Inserting the derivative above leads to the Reynolds transport theorem:
\begin{equation}
\frac{\mathrm{d}m}{\mathrm{d}t} =  \int_V \frac{\partial f }{\partial t} \mathrm{d }V + \int_V f \mathbf{v}\cdot \mathbf{n} dS
\end{equation}
Infinitesimal control volumes:
For  $V \rightarrow 0 $ the 
Conservation: 
\begin{equation}
\frac{\mathrm{d}m}{\mathrm{d}t} = 0  
\end{equation}•
translate Lagrange to Eulerian: 
\begin{equation}
\frac{\mathrm{d}m}{\mathrm{d}t}  = \int_V \frac{\partial f }{\partial t} \mathrm{d}V +  \int_V f \mathbf{v}\cdot \mathbf{n} dS
\end{equation}
With the  field $\mathbf{v}$ Fick's law is defined as:
\begin{equation}
f \cdot \mathbf{v} = -{D} \nabla f
\end{equation}
Where D is Diffusion tensor with the elements $D_{ij} ,[i,j]  \{x,y,z\}$
Insertion of the the relations derived above gives: 
\begin{equation}
\frac{\mathrm{d}m}{\mathrm{d}t}  = \int_V \left[ \frac{\partial f }{\partial t } - \nabla (D\nabla f) \right]\mathrm{d}V = 0
\end{equation}
This relation recovers Fick's second law:
\begin{equation}
\frac{\partial f}{\partial t} = D \Delta f
\end{equation}\\[1em]
Conclusion: \\
In an Eulerian description the formulation of the advective transport needs to be addressed with the term $ - \nabla \cdot (f \cdot \mathbf{v}(\mathbf{r},t))$. This can be avoided using a Largangian formalism as the convective transport is described by the movement of the control volumes  
 
\end{document} 
