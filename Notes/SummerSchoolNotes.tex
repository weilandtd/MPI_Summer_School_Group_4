\documentclass[hyperref]{labbook}

%\usepackage[utf8]{inputenc}
\usepackage[english]{babel}

\usepackage[top=1in, bottom=1in, left=1in, right=1in]{geometry}

\usepackage{amsmath}
\usepackage{graphicx}
\usepackage{epstopdf}
\usepackage{float}
\usepackage{listings}
\usepackage{color}
\usepackage{hyperref}
\usepackage{tikz}

\numberwithin{equation}{experiment}
\numberwithin{table}{experiment}
\numberwithin{figure}{experiment}

%Formatting of the Code Parts 
\definecolor{maroon}{rgb}{0.5,0,0}
\definecolor{darkgreen}{rgb}{0,0.5,0}
\lstdefinelanguage{XML}
{
  basicstyle=\ttfamily\footnotesize,
  morestring=[b]",
  moredelim=[s][\bfseries\color{cyan}]{<}{\  },
  moredelim=[s][\bfseries\color{cyan}]{</}{>},
  moredelim=[l][\bfseries\color{cyan}]{/>},
  moredelim=[l][\bfseries\color{cyan}]{>},
  morecomment=[s]{<?}{?>},
  morecomment=[s]{<!--}{-->},
  commentstyle=\color{orange},
  stringstyle=\color{red},
  identifierstyle=\color{darkgreen}
}


\begin{document}

\numberwithin{lstlisting}{experiment}
\labday{24.07.2016}
\experiment{Dynamic systems modeling}
System: a potential source o data
\begin{itemize}
\item{\color{red} boundary }
\item inputs / outputs
\end{itemize}
Experiments: Extracting data from a system 
\begin{itemize}
\item apply condition to input and observe outputs
(Obeservability / Controlability)
\end{itemize}
Model: of a system {\color{red} and an experiment is anything} to which exp can be applied.
This is results in an experimental frame of the model. A model in the general case doesn't need to be mathematical or computational. No model is valid for all experiments but the system itself.
The purpose of modeling is to simply. "All models are wrong but some of them are useful" \\[1em]
Simulation: Perform the experiment on the model. \\[1em]
Continuity: 
\begin{itemize}
\item Discrete and Deterministic  (Molecular Dynamics)
\item {\color{red} Continuous  and Deterministic (Partial differential equations)}
\item Discrete and Stochastic  (Agent based models)
\item Continuous  and Stochastic (Stochastic differential equations) 
\end{itemize}

The continuum limit assumes the changes of a single molecular does not matter for the overall concentration in the region of the length scale  $\lambda$. 
The length scale $L$ field gradients are established  \\
\begin{equation}
\mathrm{Kn} := \frac{\lambda}{L} 
\end{equation}•
If $\mathrm{Kn} << 1 $ the continuum assumption is justified. \\[1em]
Extensive quantities: depend on the system size.(Mass, Volume, molecules \dots) \\
Intensive quantities: do not depend on the system size (Temperature, concentration \dots). \\[1em]
Differential equations are always expressed in terms of intensive quantities in simulation needs to use extensive quantities. This is not necessary but helpful to design the simulation experiment properly. \newpage
Dimensional Analysis: 
\begin{itemize}
\item dependence of variables (Which are the important variables?)
\item orders of magnitude
\item Model - System similitude (Where show model and system same dynamics?)
\end{itemize}
Dim:   Mass, length, time, force,  \dots\\
Units: kg, lb \dots \\[1em]
Theorem:\\
All dimensions are power series of 6 basic dimensions.
\begin{itemize}
\item mass M
\item length L 
\item time T 
\item temperature $\theta$
\item charge C
\item resistance R
\end{itemize}
Example: $force = M^1L^1T^{-1}$\\[1em]
Buckingham Theorem:\\
If n is the number of basic dimensions  and p variables of a system. The system is completely described by p-n dimensionless groupings. \\
Example:\\
A force acting on a point mass in vacuum: \\
$n = 3 (\mathrm{M,L,T})$\\
$p = 4 (\mathrm{m,F,t,v})$\\
Therefore, the system is described by a single dimensionless quantity.\\[1em]
All physical equations are dimensionally homogeneous: $A + B = C $. The dimensions of A, B and C needs to be the same. \\[1em]
Taylor (1979) proposed and algorithm to find the dimensionless groupings of a system. \\
Example:\\
Shear stress on red blood cell traveling with a velocity $u$ though an artery. If the shear stress $\tau$ is to hight the blood cell would be destroyed. The shear stress then depends on the distance from the wall $h$, the viscosity $\eta$, the density $\rho$ and the velocity $u$.\\[1em]
$n = 3 (M,L,T)$\\
$p = 4 (\eta,\rho,h,\tau,u)$\\

\begin{equation}
\begin{matrix}
          & M & L & T \\
\hline
h       & 0  & 1 & 0 \\
u       & 0  & 1 & -1 \\
\eta & 1 & -1 & -1 \\
\rho & 1  & -3 & 0 \\
\tau & 1  & -1 & -2 \\
\end{matrix} \rightarrow
\begin{matrix}
          & M & L & T \\
\hline
h       & 0  & 1 & 0 \\
u       & 0  & 1 & -1 \\
\eta & 0 &  2 & -1 \\
\rho & 1  & -3 & 0 \\
\tau & 0  & 2 & -2 \\
\end{matrix}
\rightarrow
\begin{matrix}
           & L & T \\
\hline
h         & 1 & 0 \\
u         & 1 & -1 \\
\eta/\rho u &  1 &0 \\
\tau/\rho u^2  &0 & 0 \\
\end{matrix}
\rightarrow
\begin{matrix}
           & L  \\
\hline
\eta/\rho u h&  0 \\
\tau/\rho u^2  &0 \\
\end{matrix}
\end{equation}
The dimensionless numbers are then $\Pi_1 = \frac{\eta}{\rho u h}$ and $\Pi_1 = \frac{\tau}{\rho u^2}$.\\
We also know that: $\Pi_1 = f(\Pi_2 )$ this helps us to design an experiment because only the dimensionless numbers need to be varied to understand the function the relates  $\Pi_1$ and $\Pi_2$ \\[2em]
Modeling Dynamics:\\[0.5em]
1) Define the system boundaries and the input and outputs\\
2) Identify reservoirs (for any quantity) of relevant time scales \\
3) Formulate equations for the flows to the reservoirs. \\All flows have the form $flow = f(activating level - inhibiting level)$. \\
4) Formulate the balance equations for the reservoirs. $\frac{d}{dt}level = \sum inflows - \sum outflows $
5) Simplify the system of equations\\
6) Solving the systems will give $level(t)$\\
7) Identify unknown parameters based on the given data\\
8) {\color{red} Validate the model}. Prove that the model can predict another experiment!\\[1em]
\experiment{Recap of vector calculus}

\experiment{Spatiotemporal models}
Control volumes:\\
Discretization of space into finite small volumes. Considering that the control volumes can vary with time the rate of change of the intensive variable $f$. For being the density $f = m/V$:
\begin{equation}
\frac{df}{dt} = \frac{1}{V}\frac{dm}{dt} + \frac{m}{V}\frac{1}{V}\left( -\frac{dV}{dt}\right)
\end{equation}•\\[1em]
Fixes control volumes / Eulerian :\\
When using fixed control volumes the flow between the control volumes $V_i$ is define by the field derivative: 
 \begin{equation}
\frac{\partial f(x,t)}{\partial t }|_{x=const} = D\nabla^2f - \nabla \cdot (f \cdot \mathbf{v}(\mathbf{r},t)) + k(f)
\end{equation}
Co-moving control volume / Lagrangian: \\
When considering moving volumes the flow of the intensive quantity is given by the material derivative: 
\begin{equation}
\frac{\mathrm{d}f}{\mathrm{d}t} = \frac{\partial}{\partial t} \left[f(\mathbf{r}(r), t ) \right] = \nabla f \cdot \frac{\partial  \mathbf{r}(t)}{\partial t} + \frac{\partial f }{\partial t}
\end{equation}\\[1em]
For considering mass conservation: 
\begin{equation}
m = \int f \mathrm{d}V  
\end{equation}•
\begin{equation}
\frac{\mathrm{d}m}{\mathrm{d}t} = 0 
\end{equation}
Inserting the derivative above leads to the Reynolds transport theorem:
\begin{equation}
\frac{\mathrm{d}m}{\mathrm{d}t} =  \int_V \frac{\partial f }{\partial t} \mathrm{d }V + \int_V f \mathbf{v}\cdot \mathbf{n} dS
\end{equation}
Infinitesimal control volumes:
For  $V \rightarrow 0 $ the 
Conservation: 
\begin{equation}
\frac{\mathrm{d}m}{\mathrm{d}t} = 0  
\end{equation}•
translate Lagrange to Eulerian: 
\begin{equation}
\frac{\mathrm{d}m}{\mathrm{d}t}  = \int_V \frac{\partial f }{\partial t} \mathrm{d}V +  \int_V f \mathbf{v}\cdot \mathbf{n} dS
\end{equation}
With the  field $\mathbf{v}$ Fick's law is defined as:
\begin{equation}
f \cdot \mathbf{v} = -{D} \nabla f
\end{equation}
Where D is Diffusion tensor with the elements $D_{ij} ,[i,j]  \{x,y,z\}$
Insertion of the the relations derived above gives: 
\begin{equation}
\frac{\mathrm{d}m}{\mathrm{d}t}  = \int_V \left[ \frac{\partial f }{\partial t } - \nabla (D\nabla f) \right]\mathrm{d}V = 0
\end{equation}
This relation recovers Fick's second law:
\begin{equation}
\frac{\partial f}{\partial t} = D \Delta f
\end{equation}\\[1em]
Conclusion: \\
In an Eulerian description the formulation of the advective transport needs to be addressed with the term $ - \nabla \cdot (f \cdot \mathbf{v}(\mathbf{r},t))$. This can be avoided using a Largangian formalism as the convective transport is described by the movement of the control volumes  
 \labday{24.07.2016}
\experiment{Cellular Automata}
Simple model of complex systems: Simple rules between the finite control volumes can reproduce the complex behavior exhibited by the differential equations. The complex behavior emerges of the population of interacting control volumes. The emerging behavior is lost when isolating the components. \\[1em]
Example: Self organization of heart cells. \\[1em]
Definition of the CA: 
\begin{itemize}
\item Discrete space, lattice or cells (nodes and sites ) 
\item States: a discrete - typically small set of states that characterize the cells 
\item Neighborhood: cells that interact with the ith cell  
\item Update rules 
\end{itemize}
Types of update rules: 
\begin{itemize}
\item Deterministic (Eularian) 
\item Stochastic: Execution of the rule has a probability  
\item Lattice Gas CA for moving cells 
\item Solidification CA with absorbing state for architectural models e.g. branching morphogenesis
\end{itemize}
Example: John Conway's Game of Life\\
2D System with two sates and three rules: Birth, Survival and Death. \\
Birth: if cell is empty and has exactly three living neighbors  \\
Survival: if cell is living and has two or three living neighbors \\
Death: else ?    ... 
\newpage
\experiment{Particle Methods}
Discretization: \\
Physical quantities are usually present as continuous fields. In computer simulation the fields needs to be sampled at "representative points". The required resolution is determined by the field limit $L$ and the continuous limit $\lambda$.
\begin{equation}
\lambda < \Delta x < L  \rightarrow \Delta x < \frac{2}{a}
\end{equation}
where $a = max(Eig.val.)$. The discretization method can be structured or unstructured. Possible methods are:
\begin{itemize}
\item Finite Differences, 1D Grid
\item Finite Elements,  2D Triangles
\item Finite Volumes, 3D Volumes 
\item Particle Methods, 0D Points
\end{itemize}
Particles as reservoirs: Every particle is a reservoir and interchange its interchanging flows with it surrounding particles. \\[1em]
Particle methods:\\
In terms of discretization particle methods are very suited for the simulation of biological system because its easy to from complex geometry.\\ 
A Particle is a set $(\mathbf{r}, \{ w_1 , w_1 \dots \})$ where $\mathbf{r}$ is the position and $w_i$ are arbitrary variables. 
\begin{equation}
\frac{\mathrm{d}\mathbf{r}}{\mathrm{d}t}  =  \sum_{q = 1}^N K(\mathbf{r}, \{ w_1 , w_1 \dots \})
\end{equation}
\begin{equation}
\frac{\mathrm{d}\mathbf{w}}{\mathrm{d}t} = \sum_{q = 1}^N F(\mathbf{r}, \{ w_1 , w_1 \dots \})
\end{equation}
In General the implementation of particle methods is always the same only the functions $K$ and $F$ need to be provided. \\
Libraries: PPM, Puma .. .\\[1em]
Solve a differential equation: \\
\begin{equation}
\frac{\mathrm{d}u}{\mathrm{d}t} = f
\end{equation}
First possibility to compute the solution is discretization of the equation and then numerical integration. Therefore, the consistency is crucial for discretization and numerical stability for the actual computation. This methods is called intensive solution. 
\begin{equation}
\frac{u_n - u_{n-1}}{\delta t} = f_{n-1} \rightarrow u_n = u_{n-1} + \delta t f_{n-1}
\end{equation}
Second possibility express the differential equation as an integral and then compute this integral numerically. In contrast to the first possibility the expression of the integral is exact numerical error is only exhibited by computation of the integral (stability is not an issue). This method is called is intensive. 
\begin{equation}
u = \int f(t ) dt \rightarrow u = \sum w_i f(x_i)
\end{equation}
Numerical stability can be derived for a simple first order equation solved with the Euler scheme as :
\begin{equation}
\frac{x_n - x_{n-1}}{\delta t} = -\lambda x_{n-1}
\end{equation}
is stable for 
\begin{equation}
|1- \lambda t| < 1 \rightarrow  0 <\delta t < \frac{2}{\lambda}
\end{equation}
Particle Function Approximation: \\
\begin{equation}
u(x) = \int_\Omega u(y) \delta(x-y) dy =   \int_\Omega u(y) \xi(x-y) dy + \mathcal{O}(\epsilon^r)
\end{equation}
where $\xi$ is a smoothing function the is similar to delta 
\begin{equation}
\xi_{\epsilon} = \frac{1}{\epsilon}\xi\left(\frac{x}{\xi}\right)
\end{equation}
If we consider the space discretized with the particle positions $x_p$
\begin{equation}
u(x)  \approx  \sum_\Omega u(x_p) \xi(x-x_p) V_p = \sum_{p  = 1}^N \omega_p \xi(x-x_p)+ \mathcal{O}\left( \frac{h}{\epsilon}\right)^S
\end{equation}
Choosing $\xi$ will determine  $\mathcal{O}(\epsilon^r)$ where $r$ is the $r$th  moment that differs from the delta function. Thus for a Gaussian function $r = 2$. The error  $\mathcal{O}\left( \frac{h}{\epsilon}\right)^S$ where $S$ is the smoothness of the function defined be the $n$th defined derivative. For a Gaussian function this l eads to $S \rightarrow \infty $ therefore the discretization length $h$ of the particles is required to be smaller than $\epsilon$. This leads to an overlap of particles. This leads to smoothed particle methods. \\[1em]
Remeshig: \\
Use a gird with the a grid size $a < \epsilon$.If the the overlap condition is violated:  Interpolate the values to the mesh nodes. Delete the particles. Reinitialize particles at the mesh points. Therefore, the discretization is adaptive.\\ 
As the smoothing function was designed to conserve the first two moments, an interpolation methods is desired the conserve this moments. A interpolation scheme of an arbitrary order convolved with itself will give the interpolation scheme of the next higher order. This a methods conserving moments to the order of 3 is given by: \\ 
\begin{equation}
M'(s ) = \begin{cases}  1 - \frac{1}{2}(5s^2 -  3 s^3)  & 0\leq s \leq 1 \\ \frac{1}{2} (2-s)^2(-s) & 1 \leq s \leq 2 \\ 0 & s > 2  \end{cases} \; \text{where} \; s = \frac{|x_p - x_m|}{h}
\end{equation}
To compute the interactions in between $N$ particle pairs has a complexity of $\nu(N^2)$. But if the function $F$ only changes the particle locally the complexity reduces to $\nu(N)$. If $F$ is global a method 1984 Barnes \& Hut can compute all interaction with a complexity  $\nu(Nlog(N))$. The idea is to cluster particles into cells and compute the interaction of the particles with its nearby particle and the interaction with the clusters. Another method proposed by Greengaard \& Roklin 1989 reduces this further $\nu(N)$. The Greengaard \& Roklin let only representative cluster interact with each other.\\[1em]
Cell List: \\
Create a grid with the grid size of the cutoff radius $r_c$. You loop over every particle and you get the index of the cell $i = Int(x_p/r_c)$. In a next step the particle is added to the list of this cells. To read out the nearest neighbors of the particle all particles in the particles cell and and the cells neighbors.\\[1em]
Verlet list: \\ 
Every particle caries a list of interacting particles. Therefore, the Verlet list  uses a slightly larger cutoff radius than the cell list it is created from. As long as no particle moves in this skin the same Verlet list can be used for the particle and the list doesn't need to be constructed. \\[1em]
\experiment{Simulation of Diffusion}
Mesoscopic picture: Consider microscopic phenomena on a macroscopic length scale. We consider the probability of a particle to change its position with in the time interval $\delta t$. The probability density function of the particle to be a the position $x$ is then: 
\begin{equation}
P(x,t+\delta t ) )  = q_{+} P(x-\delta x , t) + q_{-} P(x+\delta x , t)+P(x,t)[1-(q_{-}+q_{+}  )]
\end{equation}
where $q$ is the transition property and P is a probability density function. Using the second order Taylor expansion:
\begin{equation}
 P(x,t) + \delta t \frac{\partial P }{\partial t} +\frac{ \delta t ^2 }{2!} \frac{\partial^2 P}{\partial t ^2} + \mathcal{O}(\delta t ^3) =  q_{+} \left[  P(x , t)  - \delta x  \frac{\partial p}{\partial }    \right]+ q_{-} P(x+\delta x , t)+P(x,t)[1-(q_{-}+q_{+}  )] 
\end{equation}•
For $\delta t \rightarrow 0$
\begin{equation}
\frac{\partial P}{\partial t} = \frac{\partial P}{\partial x} \left [ \frac{q_{-}  \delta x - q_{+} \delta x  }{\delta t } \right] + \frac{\partial ^2P}{\partial^2x} \left [ \frac{\delta x^2}{2}\frac{q_{-}  + q_{+} \delta  }{\delta t } \right]+\mathcal{O}(\delta t ^3)
\end{equation}
Partial stochastic equation: 
\begin{equation}
\frac{\partial P}{\partial t } = V \frac{\partial P}{\partial x }  +D \frac{\partial^2 P}{\partial t^2 } 
\end{equation}
A solution for the probability density is : 
\begin{equation}
P(x|x_0, \delta t ) = \frac{1}{\sqrt{4\pi D \delta t}}e^{\frac{-(x-x_0)^2}{4D\delta t}}
\end{equation}
With the mean squared displacement of a particles can then be described as: 
\begin{equation}
\langle x^2(t) \rangle = 2Dt 
\end{equation}
The results can also be recovered for the random walk: 
\begin{equation}
x(t + \delta t ) = x(t) + r \sqrt{\delta t} ; r ~ N(0,2dD)
\end{equation}
Random walk Particle method: \\
Consider a particle with its position $x_p$ and the number of molecules on the particle $w_p = u^0(x_p) V_p $
\begin{equation}
x_p (t_i+ \Delta t ) \leftarrow x_p(t_i) + \mathbf{r} \sqrt{\Delta t}
\end{equation}
Using the random walk the diffusion equation is sufficiently described therefore:
\begin{equation}
\frac{\mathrm{d} w_p}{\mathrm{d}t} = 0
\end{equation}
Continuum:\\
In a continuous description diffusion is described using the densities $u(x,t) \rightarrow$ Density or Concentration. The development in time can be recovered from Fick's law:
\begin{equation}
\frac{\partial u}{\partial t} = D \nabla^2 u 
\end{equation}
Where $D$ is not a function of space and time, and the fluid is isotropic and homogeneous. \\
For the 1D diffusion the particle method for a contentious particle method (Particle Strength Exchange) is recovered from the PDE:
\begin{equation}
\frac{\partial u}{\partial t} = D \frac{\partial^2 u}{\partial x^2} 
\end{equation}
Expanding $u(x)$ with the taylor expansion: 
\begin{equation}
u(y) = u(x) + \frac{y-x}{1!}  \frac{\partial u}{\partial x} +  \frac{(y-x)^2}{2!} \frac{\partial^2 u}{\partial x^2}  + \dots
\end{equation}
\begin{equation}
\int_\Omega [u(y) - u(x)] \eta(y-x) dy = \int_\Omega \frac{y-x}{1!}  \frac{\partial u}{\partial x} \eta(y-x) dy  +   \int_\Omega\frac{(y-x)^2}{2!} \frac{\partial^2 u}{\partial x^2}  \eta(y-x) dy + \dots
\end{equation}
The Kernel  $\eta_\epsilon(\xi) = \frac{1}{\epsilon}\eta\left(\frac{\xi}{\epsilon}\right) $ shall be the delta function for $\epsilon \rightarrow 0 $
\begin{equation}
\int_\Omega [u(y) - u(x)] \eta(\xi) d\xi = \int_\Omega \frac{\xi}{\epsilon}  \frac{\partial u}{\partial x} \eta(\xi) d\xi  +   \int_\Omega\frac{\xi^2}{2\epsilon} \frac{\partial^2 u}{\partial x^2}  \eta(\xi) d\xi + \dots
\end{equation}
For the construction of the Kernel we choose: 
\begin{equation}
 \int \xi \eta(\xi) d\xi = 0 
\end{equation}
\begin{equation}
 \int \xi^2 \eta(\xi) d\xi = 2
\end{equation}
\begin{equation}
 \int \xi^s \eta(\xi) d\xi = 0; 3 \leq s \leq r +1 
\end{equation}
The gradient of $u(x)$ is then:
\begin{equation}
\frac{\partial^2 u}{\partial x^2}  \approx \frac{1}{\epsilon} \int_\omega [u(y) - u(x)] \eta_\epsilon(y-x) dy + \mathcal{O}(\epsilon^r)
\end{equation}
In terms of particles this is written as a sum: 
\begin{equation}
\frac{\partial^2 u_p}{\partial x^2}  = \frac{1}{\epsilon^2} \sum_{q=1}^N V_q [u_q - u_p] \eta_\epsilon(y-x) 
\end{equation}
In general the derivative of the order $\beta$ is given by:
\begin{equation}
\frac{\partial^\beta u_p}{\partial x^\beta}  = \frac{1}{\epsilon^\beta} \sum_{q=1}^N V_q [u_q \pm u_p] \eta^\beta_\epsilon(y-x) 
\end{equation}
A solution for the second derivative the one dimensional Kernel is 
\begin{equation}
\eta (\xi) = \frac{1}{2\sqrt{\pi}}e^{-\frac{xi^2}{4}}
\end{equation}•
A solution for the second derivative the three dimensional Kernel is 
\begin{equation}
\eta (\xi = \{x,y,z\}) = \frac{1}{(2\pi)^{3/2}}e^{-\frac{xi^2}{2}}
\end{equation}
Diffusion and Advection: \\[1em]
The general equation considering advective and diffusive transport is:
\begin{equation}
\frac{\partial u}{\partial t} + \nabla \cdot(u V) = D\nabla^2u
\end{equation}
For incompressible fluids $\nabla \cdot V = 0$ holds. Thus the equation simplfies to 
\begin{equation}
\frac{\partial u}{\partial t} + V \cdot (\nabla u) = D\nabla^2u -  u (\nabla\cdot V)
\end{equation}
To derive the particle method it is required to express it in an Lagrangian framework:
\begin{equation}
\frac{dx_p}{dt} = V_p(x_p, t )
\end{equation}•
\begin{equation}
\frac{du_p}{dt} = D\nabla^2 u_p - u_p(\nabla \cdot V) 
\end{equation}
\begin{equation}
\frac{du_p}{dt} \approx \frac{D}{\epsilon^2} \sum_{q  = 1 }^N V_p(u_q - u_p) \eta_\epsilon(x_q - x_p) - u_p(\nabla \cdot V)_p
\end{equation}
For a two dimensions the system writes: 
\begin{equation}
(\nabla \cdot V)_p  = \frac{1}{\epsilon} \sum_{q = 1 }^N \Omega_q (V_{xq} + V_{xp}) \eta^{(1,0)}(x_p - x_q) + \frac{1}{\epsilon} \sum_{q = 1 }^N \Omega_q (V_{yq} + V_{yp}) \eta^{(0,1)}(y_p - y_q) + \mathcal{O}(\epsilon^3)
\end{equation}
With the kernels: 
\begin{equation}
\eta_\epsilon^{(1,0)}(x,y) = \frac{-2x}{\epsilon^2 \pi }e^{\frac{x^2+y^2}{\epsilon^2}}
\end{equation}•
\begin{equation}
\eta_\epsilon^{(0,1)}(x,y) = \frac{-2y}{\epsilon^2 \pi }e^{\frac{x^2+y^2}{\epsilon^2}}
\end{equation}
\begin{equation}
\frac{dx_p}{dt} = V_p(x_p,t)
\end{equation}
To solve the equations in time a first order explicit Euler scheme can be implemented according to: 
\begin{equation}
x_p(t + \delta t ) = x_p(t) + \delta t V_p(x_p,t) + \mathcal{O}(\delta t^2)
\end{equation}
\begin{equation}
u_p(t + \delta t ) = u_p(t) + \delta t ( \dots) + \mathcal{O}(\delta t^2)
\end{equation}
The maximal timestep $\delta t < c||\boldsymbol\nabla\boldsymbol V||_\infty^{-1} $ or also $\delta t < c \delta x /V$\\[1em]
Chemical Reactions: \\
Consider the reaction Network: \\
\begin{equation}
\phi \rightarrow A 
\end{equation}
\begin{equation}
A+B \rightarrow C 
\end{equation}
\begin{equation}
C \rightarrow \phi 
\end{equation}
With the rate constants $R_1$, $R_2$ and $R_3$ the reaction kinetics are described by: 
\begin{equation}
\frac{d[A]}{dt} = R_1 - R_2 [A][B]
\end{equation}•
\begin{equation}
\frac{d[B]}{dt} = - R_2 [A][B]
\end{equation}•
\begin{equation}
\frac{d[C]}{dt} = R_2 [A][B] - R_3 [C]
\end{equation}•
Using the explicit Euler scheme the evolution of the concentration of the concentration is given by:
\begin{equation}
[A](t+\delta t ) = [A](t) + (k_1- k_2 [A](t)[B](t)) \delta t
\end{equation}• 
Diffusion - Advection - Reaction Equation :\\
Assume three chemical species $S_1$ , $S_2$ ,$S_3$, the system also exhibits a flow field $V$, the species have the diffusion constants $D_1$, $D_2$ and $D_3$. Assume the reaction network of \begin{equation}
\phi \rightarrow S_1
\end{equation}
\begin{equation}
S_1 + S_2 \rightarrow S_3
\end{equation}
\begin{equation}
S_3 \rightarrow \phi 
\end{equation}
The general system is descibed by:
\begin{equation}
\frac{du_1}{dt} + \nabla\cdot(u_1 V)  = D_1\nabla^2u_1 + k_1 - k_2u_1u_2
\end{equation}
\begin{equation}
\frac{du_2}{dt} + \nabla\cdot(u_2 V)  = D_2\nabla^2u_2 - k_2u_1u_2
\end{equation}
\begin{equation}
\frac{du_2}{dt} + \nabla\cdot(u_3 V)  = D_3\nabla^2u_2 + k_2u_1u_2- k_3u_3
\end{equation}
A particle is now by described by its position $x_p$ and the three concentrations $u_1$, $u_2$ and $u_3$. In the lagrangian framework the equations read then: 
\begin{equation}
\frac{dx_p}{dt} = V_p 
\end{equation}•
\begin{equation}
\frac{du_{1p}}{dt} = D_1\nabla^2u_{1p}- u_{1p}(\nabla\cdot V_p)   + k_1 - k_2u_{1p}u_{2p}
\end{equation}
\begin{equation}
\frac{du_{2p}}{dt} = D_2\nabla^2u_{2p} - u_{2p}(\nabla\cdot V_p)    - k_2u_{1p}u_{2p}
\end{equation}
\begin{equation}
\frac{du_{3p}}{dt} = D_3\nabla^2u_{3p} - u_{4p}(\nabla\cdot V_p)    + k_2u_{1p}u_{2p}- k_3u_{3p}
\end{equation}
\labday{26.07.2016}
\experiment{Stability analysis}
1. Temporal (ODE) models
\begin{equation}
\frac{d}{dt} \mathbf u(t) = \mathbf f ( \mathbf u (t) , k_i)
\end{equation}
Nullclines also called stationary states. Stable steady state with an unstable environment leads to oscillations. 
A stationary state is found be setting all the fluxed to zero:
\begin{equation}
\frac{d}{dt}  u*(t) = 0 = \mathbf f (  u* (t) , k_i)
\end{equation}
With that we can construct the functions $f_1$ and $f_2$ in the form that $u^*_1 = f_1(u_2^*)$. The steady states at a given $u_i$ is then given by the intersections of the functions $f_1$ and $f_2$. The steady states for $u_2$ are found for a given parameter set. If the number of steady state change on the choice of parameters this is called bifurcation. \\
To find the steady state the Jakobi-Matrix $J$ is derived for the function $\mathbf f(\mathbf u (t), k_i)$. In an fist order approximation the function $\mathbf f $ can then be expressen as:
\begin{equation}
 \mathbf f(\mathbf u (t), k_i) =  \mathbf f( \mathbf u*)  + J\cdot(\mathbf u (t) - \mathbf u*) + \mathcal O ((\mathbf u(t) - \mathbf u*)^2)
\end{equation}•
 \begin{equation}
 \frac{d}{dt} (\mathbf u(t) - \mathbf u*)=  \mathbf f( \mathbf u*)  + J\cdot(\mathbf u (t) - \mathbf u*) + 
\end{equation}
With this we can use the general Ansatz  for a linear equation with the  eigenvalues $\lambda_i(k_i)$ of Jacobian $J$.
\begin{equation}
\mathbf u(t) - \mathbf u* = \sum_iA_i e^{\lambda_i t}
\end{equation}•
Stability of $\mathbf u*$ is given if all $\mathrm{Re}[\lambda_i]<0$. The bifurcation points are given for $\mathrm{Re}[\lambda_j]=0$ for one j. \\[1em]
2. Spatiotemporal (PDE) models \\
Consider a one dimensional system with a constant velocity $\mathbf v (\mathbf x ,t ) = v = const.$ the transport equations read then: 
\begin{equation}
\frac{\partial u(x,t)}{\partial t} = f(u) + D\frac{\partial^2 u(x,t)}{\partial x^2} - v   \frac{\partial u(x,t)}{\partial x}
\end{equation}
The steady steady state $u*$ is assumed to be uniform ${\partial u}{\partial t} = 0$ and ${\partial u}{\partial x} = 0$. In terms of the Taylor expansion $f(u)$ is expressed as:
\begin{equation}
f(u) = f(u*) + f_u|_{u*} \cdot (u-u*) + \mathcal{O}((u-u*)^2)
\end{equation}
Thus a gain the rate of change near $u*$ is given by:
\begin{equation}
\frac{\partial (u-u*)}{\partial t} = f_u(u-u*) + D\frac{\partial^2 (u-u*)}{\partial x^2} - v   \frac{\partial (u-u*)}{\partial x}
\end{equation}
Using a Fourier transform and an the trivial ansatz $u-u*$ reads
\begin{equation}
u-u* = \int_{-\infty}^{\infty} dk a_k e^{ikx} e^{\lambda_k t}
\end{equation}•
Insertion into the PDE leads to:
\begin{equation}
\lambda_k \int dk a_k e^{ikx} e^{\lambda_k t} = f_u \cdot \int dk a_k e^{ikx} e^{\lambda_k t}  + D(ik)^2\int dk a_k e^{ikx} e^{\lambda_k t} - V(ik)\int dk a_k e^{ikx} e^{\lambda_k t} 
\end{equation}•
Integration of $\int\dots dx  e^{ik'x}$  leads to $\delta(k-k')$ therefore 
\begin{equation}
a_k \lambda_k e^{\lambda_kt} = f_u a_k e^{\lambda_kt} + D(-k^2)a_k e^{\lambda_kt} - V(ik)a_k  e^{\lambda_kt} 
\end{equation}•
\begin{equation}
\lambda_k  = f_u  + D(-k^2) - V(ik)
\end{equation}•
For the linearized equations we obtain then the solution: 
\begin{equation}
u(x,t) = u* + \int a_k e^{ik(x-vt)}e^{(f_u-Dk^2)t}
\end{equation}•
From the solution we can conclude that spatial patterns induced travel in space with the flow flied $x-vt$ and depending on the function $f_u$ and the diffusion these patterns decay of $f_u -Dk^2 < 0$ 
\end{document} 


